\section{Results}
\subsection*{\textbf{1b):} Developing your code and \textbf{1d):} A better statistical analysis}

The results of the simulations are included in the appendix as figures \ref{fig:1b_1} \ref{fig:1b_10}, \ref{fig:1b_100} and \ref{fig:1b_500}. The results were produced with the code in the repository\footnote{at commit \lstinline{b0ff612bc6666335b106af5e22a7a13a13c7cff7}}. Errors in the expectation value for the energy in the plots are computed with the blocking method\footnote{the blocking implementation is a copy of the code in Ref. \cite{ComphysGit}}. 
The exact answer for the minimum of the expecation value of the energy can be derived quote easily from \ref{eq:el_ni}. The minimum of the local energy has to be where the kinetic and potential energies cancel exactly. In the case of the 3d particle we see the following after imposing natural units. 

\begin{align}
\mathfrak{min}(E_L)  \implies \left(\frac{1}{2}\right)[4\alpha^2 (x^2 + y^2 + \beta^2 z^2) ] &= \frac{1}{2} (x^2 + y^2 + \beta z^2) \\
 2 \alpha ^2  &= \frac{1}{2} \\
 \alpha &= 0.5
\end{align}
With $\alpha$ at the minimum we then expect the following values for $\expect{E}$
\begin{equation}
\begin{split}
\text{1D: }E_L |_{\alpha = 0.5} &= \sum_i^N \alpha = N_ \alpha,\\
\text{2D: }E_L|_{\alpha = 0.5} &= \sum_i^N  2\alpha = 2N \alpha \\
\text{3D: }E_L|_{\alpha = 0.5} &= \sum_i^N  2\alpha + \alpha \beta  = 3N\alpha
\end{split}
\end{equation}
In general the energy in the non-interactive case at the variational minimum for our trial wavefunction can then be expressed as 
\begin{equation}
\mathfrak{min}(\expect{E[\alpha]}) = 0.5 \cdot N \cdot D  
\end{equation}
Where $N$ is the number of particles and $D$ is their dimension. From the figures referenced in the beginning of this section it is clear that both the numerical and analytic solutions converge to the correct solution.

\subsection*{\textbf{1e:} The repulsive interaction}

\begin{table}[]
	\centering
	\begin{tabular}{|c|c|c|c|c|c|}
		\hline
		$N$ & $\alpha$ & $\sigma^2$ & $\langle E_L \rangle$ & $\frac{\langle E_L \rangle}{N}$ & CPU-time [sek] \\
		\hline
		10 & 0.480013 & 1.73405e-06 & 21.643 & 2.1643 & 12.31 \\
		\hline
		50 & 0.480127 & 1.75833e-06 & 111.066 & 2.2213 & 1253.91\\
		\hline
		100 & 0.480279 & 1.75902e-06 & 228.682 & 2.2868 & 9854.01 \\
		\hline
	\end{tabular}
	\caption{VMC with Metropolis sampling. Calculations done in 3 dimensions with number of cycles $N_{MC} = 10^{6}$ for an elliptical trap, $\beta = 2.82843$ with interaction, $a = 0.0043$. Step length $t = 0.5$.}
	\label{tab:Re.int.}
\end{table}

Comment to be added to discussion: we see that the expectation value of the local energy increases by a small amount. This is consistent with what we see in Fig. 2 in (ref. to paper in Physics by J.L. DuBois). 

\subsection*{\textbf{1f:} Finding the best parameter}
Applying the gradient descent method to tune the variational parameter resulted in satisfying convergence to the optimal value for the variational parameter $\alpha$. The simulation results are included in figure \ref{fig:gd_nm_nia}\footnote{The simulations were produced with the file \lstinline{mainf_f.cpp} at commit \lstinline{2d03030ec5a711383d582dde23869483d122c6b6}}


\begin{figure}
\includegraphics{figures/GD_NM_NIA.pdf}
\caption{Finding the optimal $\alpha$ with GD. This plot show the number of iterations it took to reach the minimum for the non-interactive spherical trap.}\label{fig:gd_nm_nia}
\end{figure} 