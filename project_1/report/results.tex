\section{Results}
\subsection*{\textbf{1b):} Developing your code and \textbf{1d):} A better statistical analysis}

The results of the simulations are included in the appendix as figures \ref{fig:1b_1} \ref{fig:1b_10}, \ref{fig:1b_100} and \ref{fig:1b_500}. The results were produced with the code in the repository\footnote{at commit \lstinline{b0ff612bc6666335b106af5e22a7a13a13c7cff7}}. Errors in the expectation value for the energy in the plots are computed with the blocking method\footnote{the implementation is a copy of the code from LINK TO THE GITHUB WITH MARIUS'S CODE}. 
The exact answer for the minimum of the expecation value of the energy can be derived quote easily from \ref{eq:el_ni}. The minimum of the local energy has to be where the kinetic and potential energies cancel exactly. In the case of the 3d particle we see the following after imposing natural units. 

\begin{align}
\mathfrak{min}(E_L)  \implies \left(\frac{1}{2}\right)[4\alpha^2 (x^2 + y^2 + \beta^2 z^2) ] &= \frac{1}{2} (x^2 + y^2 + \beta z^2) \\
 2 \alpha ^2  &= \frac{1}{2} \\
 \alpha &= 0.5
\end{align}
With $\alpha$ at the minimum we then expect the following values for $\expect{E}$
\begin{equation}
\begin{split}
\text{1D: }E_L |_{\alpha = 0.5} &= \sum_i^N \alpha = N_ \alpha,\\
\text{2D: }E_L|_{\alpha = 0.5} &= \sum_i^N  2\alpha = 2N \alpha \\
\text{3D: }E_L|_{\alpha = 0.5} &= \sum_i^N  2\alpha + \alpha \beta  = 3N\alpha
\end{split}
\end{equation}
In general the energy in the non-interactive case at the variational minimum for our trial wavefunction can then be expressed as 
\begin{equation}
\mathfrak{min}(\expect{E[\alpha]}) = 0.5 \cdot N \cdot D  
\end{equation}
Where $N$ is the number of particles and $D$ is their dimension. From the figures referenced in the beginning of this section it is clear that both the numerical and analytic solutions converge to the correct solution.