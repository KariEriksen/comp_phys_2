\section{Theory}

\subsection{Ansatz for the wavefunction and the hamiltonian}

In this project we study the physics of a dilute many boson system in a Bose Einstein condensate with variational monte carlo methods. The particles are placed in a harmonic oscillator trap described by equation 1 in the project description. We'll start by discussing a hamiltonian in which there is no interaction between particles and being bosons there are no restrictions on the occupancy of a single energy state. But in general the trial wavefunction (or ansatz for the wavefunction) is defined as in equation \ref{eq:psi_t}

\begin{equation}\label{eq:psi_t}
\Psi_T(\{\mathbf{r}\}) = \prod _i g(\alpha, \beta, \mathbf{r}_i) \prod _{i <j} f(a, |\mathbf{r}_i -\mathbf{r}_j|)
\end{equation} 
\noindent Where $g(\alpha, \beta, \mathbf{r}_i)$ is a function proportional to the single particle solution of the harmonic oscillator trap. Which is to say it can be written on the form 

\begin{equation}
g(\alpha, \beta, \mathbf{r}_i) = e^{-\alpha ( x_i^2+ y_i^2 + \beta z_i ^2)}
\end{equation}


\noindent Where $\alpha$ is the variational parameter of interest. Which is to say that minimizing the system energy $\langle \Psi_T(\{\mathbf{r}\}) | \hat{H} |\Psi_T(\{\mathbf{r}\}) \rangle$ by $\alpha$ is the objective of this project. 

In equation \ref{eq:psi_t} we also define the correlation wavefunction $f(a, |\mathbf{r}_i -\mathbf{r}_j|)$ is  defined as 

\begin{equation}
f(a, |\mathbf{r}_i -\mathbf{r}_j|) = 
\begin{cases}
\infty, & |\mathbf{r}_i -\mathbf{r}_j| \leq a \\
0, & |\mathbf{r}_i -\mathbf{r}_j| > a
\end{cases}
\end{equation}

\noindent Initially (for tasks a, b, and c) we set the parameter a to equal zero and solve the non-interactive system. As it turns out directly evaluating the inner product to find the expectation for the energy is problematic because for a system with interaction it involves solving a pretty unasvory integral. We then turn to numerical methods as a tool, but we still want avoid directly evaluating the expectation for the hamiltonian for the ensamble. The solution to this problem becomes appernt when you consider the machinery of the VMC.

\subsection{Variational Monte Carlo}

Monte Carlo simulations are widely used methods in numerical science that employs random walkers. 
In this project we are taking a closer look at trapped bosons. We are given a trail wave function we assume is as close to the real case as possible, $\Psi_T(\mathbf{R};\alpha)$ where $\mathbf{R} = (\mathbf{R}_1, ... , \mathbf{R}_N)$ is the position of the different particles. 
From quantum mechanics we know the probability distribution is given by the wave function. 

$$P(\mathbf{R}; \alpha) = \frac{|\Psi_T(\mathbf{R};\alpha)|^2}{\int |\Psi_T(\mathbf{R};\alpha)|^2 d\mathbf{R}}$$

Monte Carlo integration allow us to evaluate the integral at hand. The expectation value of the Hamiltonian is given as follows. 

$$\langle \widehat{\mathbf{H}}\rangle = \frac{\int d \mathbf{R} \Psi^{\ast} (\mathbf{R})H(\mathbf{R}) \Psi(\mathbf{R})}{\int d \mathbf{R} \Psi^{\ast} (\mathbf{R}) \Psi(\mathbf{R})}$$

The variational principle states that the expectation value of the Hamiltonian is an upper-bound for the ground state energy of the Hamiltonian.

$$E_0 \leq \langle H \rangle$$

This is what the Variational Monte Carlo method bases itself on. Given a probability distribution we can evaluate the wave function and look for a local minimum. We define the local energy by

$$\widehat{\mathbf{E}}_L(\mathbf{R};\alpha) = \frac{1}{\Psi_T(\mathbf{R};\alpha)}\widehat{\mathbf{H}}\Psi_T(\mathbf{R};\alpha).$$

Then the expectation value of the local energy is given by

$$\langle \widehat{\mathbf{E}}_L\rangle = \int P(\mathbf{R}) \widehat{\mathbf{E}}_L d\mathbf{R} \approx \frac{1}{N} \sum_{i = 1}^N \mathbf{E_L}(x_i)$$

where $N$ is the number of Monte Carlo cycles. Now we can calculate the probability distribution and the local energy. And for each cycle we propose a new configuration $\mathbf{R}_p$ for the system and hopefully we come closer to the ground state.

$$\mathbf{R}_p = \mathbf{R} + r \ast \Delta \mathbf{R}$$

\subsection{The Metropolis Algorithm}

How we select the new configurations during the simulation is given by the Metropolis Algorithm. 
Ideally we would have a transition probability matrix which told us how likely it would be for each particle to move in the configuration space. Since we don't we try and model it.
We define the following entities:
\\
\\
$P_i^{(n)} \rightarrow$ probability of finding the system in state $i$ at the n'th step.
\\
$j \rightarrow$ possible new step
\\
$A_{i \rightarrow j} \rightarrow$ probability of acceptance. 
\\
$T_{j \rightarrow i} \rightarrow$ probability of making the transition  
\\
\\
Now we can say that a transition probability matrix can be constructed by $T_{j \rightarrow i}A_{j \rightarrow i}.$ The probability of finding the system in state $i$ at step $n$ is

\begin{equation}\label{eq:prob}
P_i^{(n)} = \sum_j \left[P_j^{(n-1)} T_{j \rightarrow i}A_{j \rightarrow i} + P_j^{(n-1)}T_{i \rightarrow j}(1 - A_{i \rightarrow j})\right].
\end{equation}

We want to push the system towards high density regional space of $P_i^{(n)}$. /We want to select states according to the probability distribution. In that way to converge to the desired stationary distribution $p_i$.

$$P_i^{(n \rightarrow \infty)} \rightarrow p_i$$

Using this statement and the fact that the sum over all possible transitional probabilities is one we can rewrite the above equation \ref{eq:prob}.

$$\sum_j [p_j T_{j \rightarrow i}A_{j \rightarrow i} - p_i T_{i \rightarrow j}A_{i \rightarrow j}] = 0$$

In order to stop at the configuration where we have reached equilibrium we us the condition of detailed balance, and it gives us 

\begin{algorithm}
\KwData{matrix $\mathbf{R}$}
\KwResult{float $\mathbf{E}_L$}
number of MC-cycles $N$\;
initialize $\mathbf{R}$\;
calculate $|\Psi_T(\mathbf{R})|^2$\;
\For{i in $[0, N]$}{
    $\mathbf{R}_p = \mathbf{R} + r \ast \Delta \mathbf{R}$\;
    calculate $\omega = \frac{|\Psi_T(\mathbf{R}_p)|^2}{|\Psi_T(\mathbf{R})|^2}$\;
    \eIf{$q \leq \omega$}{
        accept new move\;
    }
    {reject new move\;
    }
    update energy, $\mathbf{E}_L$\;
}
 \caption{Monte Carlo with Metropolis-Hastings}\label{alg:metro}
\end{algorithm}

$$\frac{A_{j \rightarrow i}}{A_{i \rightarrow j}} = \frac{p_i T_{i \rightarrow j}}{p_j T_{j \rightarrow i}}.$$

We choose to accept when the acceptance is bigger than 1. 

$$A_{j \rightarrow i} = \mathrm{min} \left( 1, \frac{p_i T_{i \rightarrow j}}{p_j T_{j \rightarrow i}}\right)$$

What we must calculate is equation \ref{eq:ratio}. And an example of the code is given further down, in Algorithm \ref{alg:metro}

\begin{equation}\label{eq:ratio}
\frac{p_i T_{i \rightarrow j}}{p_j T_{j \rightarrow i}} = \frac{|\Psi_T(\mathbf{R}_p)|^2}{|\Psi_T(\mathbf{R})|^2}
\end{equation}

\subsection{Importance Sampling}

Now we move to an extension of the Metropolis algorithm, the Importance sampling. This method provides us with a better way of suggesting new moves. The expression for the new move, $y$, is given by the solution of the Langevin equation. It reads 

$$y = x + D F(x) \Delta t + \xi \sqrt{\Delta t},$$

where $\xi$ is a gaussian random variable, $D$ is the diffusion coefficient and is $\frac{1}{2}$ and $\Delta t$ is the time step which take values between $[0.001, 0.01]$.
$F(x)$ is the drift force and is given by the gradient of the wave function. 

$$F = 2 \frac{1}{\Psi_T} \nabla \Psi_T$$

It is the drift force that ensures us we move particles towards regions of configuration space where the trail wave function is large. This method increases the efficiency of our program, since the standard Metropolis can suggest new moves in every direction with same probability. 
From the solution of the Fokker-Planck equation we get a new transition probability, the Green's function. 

$$G(y, x, \Delta t) = \frac{1}{(4 \pi D \Delta t)^{3N/2}} \exp (-(y - x - D F(x) \Delta t)^2/4 D \Delta t)$$

Now equation \ref{eq:ratio} from Metropolis algorithm changes to 

$$\frac{p_i T_{i \rightarrow j}}{p_j T_{j \rightarrow i}} = \frac{G(x, y, \Delta t)|\Psi_T(y)|^2}{G(y, x, \Delta t)|\Psi_T(x)|^2},$$

and we must calculate the drift force in order to find new moves and the Green's function to accept/reject these moves.

\subsection{Gradient Descent}

Like most problems in physics the variational quantum problem is one of optimization. In this project we are optimizing the expectation value of the energy over the parameter $\alpha$. In the naive implementation we simply span a reasonable range of $\alpha$ \footnote{an implementation of which can be seen in the file \lstinline{main_b.cpp}}. For a multi-parameter variational problem this approach is too computationally expensive to implement in most cases. To remedy this problem iterative gradient methods have been developed to make more intelligent choices for the variational parameters. In this project we have opted to implement a simple gradient descent method. The gradient descent is ordinarily presented as 

\begin{equation}
\alpha_{i+1} = \alpha_i -\gamma \frac{\partial \langle E_L (\alpha_i) \rangle}{\partial \alpha_i}
\end{equation}

\noindent Where we have opted to update the parameter $\gamma$ iteratively by the Barzilai-Borwein formula which is a method that uses an iterative approach to the step size $\gamma$ derived in much the same manner as for quasi-Newton methods\cite{OptTeory}. The only constraint we have to enforce is that $0 < \alpha$. In the algorithm we have a simple rule  that checks the sign of $\alpha$ and makes a correction if it is out-of-bounds. For $i > 0$ the step size $\gamma_i$ is then given as: 

\begin{equation}
\gamma_i = (\alpha_{i+1} - \alpha_i) \cdot \frac{1}{\expect{E_L}^\alpha_{i+1} - \expect{E_L}^\alpha_i}
\end{equation}

\noindent We also introduce the notation $\frac{\partial \langle E_L^i \rangle}{\partial \alpha} = \expect{E_L}^\alpha_{i}$ for brevity, and the same notation will be used for the tiral wavefunction $\psi_T$. The partial derivative itself can be found by the equation\footnote{The equation was retrieved from the lecture notes} 

\begin{equation}
\expect{E_l}^\alpha_{i} = 2 \left(\expect{\frac{\psi^\alpha}{\psi} E_L}^i - \expect{\frac{\psi^\alpha}{\psi}}\expect{E_L}^i \right)
\end{equation} 

\noindent Wherein the fraction of the partial derivative of the wavefunction by the wavefunction must be derived, which is trivial with the ansatz of the wavefunction. 


\begin{align}
\frac{\psi^\alpha}{\psi} &= \frac{\psi \prod_i^N (x_i^2 + y_i^2 + \beta z^2_i)}{\psi} \\
&= \prod_i^N (x_i^2 + y_i^2 + \beta z^2_i)
\end{align}
\noindent This holds for both the interactive and non-interactive case since the correlation wavefunction, $f(r_i, r_j)$, does not have any dependence on $\alpha$\\
In the program the only addition to make is the computation of two additional expecation values\footnote{implemented in \lstinline{vmc.cpp}}. Empirically we've also seen that we can reduce the number of MC-cycles quite drastically and still get a good convergence to the optimal $\alpha$. In summary gradient descent and other associated methods allows for a radical speedup in estimation of the optimal values of the variational parameters. The implementation of this method can be found in the file \lstinline{main_f.cpp}.


\subsection{Blocking Method}

For the statistical analysis and error estimates we have used the technique of blocking. \
We are looking for the expectation value of the ground state energy of our system. In order to say something about how accurate these results are we want to look at the standard deviation. 

If our samples where uncorrelated we could calculate the standard deviation through the following equation 
$$\sigma = \sqrt{\frac{1}{n - 1} \left(\langle E_L^2\rangle - \langle E_L \rangle^2\right)}.$$

It can be showed that for correlated samples the equation for the standard deviation is
$$\sigma = \sqrt{\frac{1 + 2 \tau / \Delta t}{n - 1} \left(\langle E_L^2\rangle - \langle E_L \rangle^2\right)}$$

where $\tau$ is the time between one sample and the next uncorrelated sample and $\Delta t$ is the time between each sample.\
If we knew what $\tau$ was we could simply find our $\sigma$, but since we don't we use blocking to find it. The method is simple enough, we divide our samples of data into blocks and calculating the mean of each block. So if we have an amount of samples of $\langle E_L \rangle$, we divide these into $M$ blocks and calculate the mean. By plotting the standard deviation as function of the block size we can keep blocking and blocking until we see that the std. dev. stops increasing, and this is where the blocks are uncorrelated. Now we have an estimate for $\tau$ and thus the standard deviation.


\subsection{Numerical differentiation}
To evaluate the local energy of the system as defined in the project it necessary to compute the second derivative of the trial wavefunction $\psi_T$. We've chosen to implement the numerical differentiation as a finite difference approximation. Let $\mathbf{R}$ be the row major $N \times D$ matrix where $N$ is the number of particles and $D$ is their dimension. Then the second derivative can be found by the procedure listed as algorithm \ref{alg:nd}

\begin{algorithm}
\KwData{matrix $\mathbf{R}$}
\KwResult{float $\nabla ^2 \psi_T (R)$}
\BlankLine
$\Delta = 0$ \\
$\mathbf{R_p} = \mathbf{R}$\\
$\mathbf{R_m} = \mathbf{R}$\\
\BlankLine
\For{$i$ in $[0, N-1]$}{
	\For{$j$ in $[0, D-1]$}{
		$\mathbf{R_p}(i,j) += h$ \\
		$\mathbf{R_m}(i,j) -= h$ \\
		$\Delta = \psi_T(\mathbf{R_p}) + \psi_T(\mathbf{R_m}) - 2 \psi_T(\mathbf{R})$\\

		$\mathbf{R_p}(i,j) -= h$ \\
		$\mathbf{R_m}(i,j) += h$ \\
	}
}
\KwRet{$\frac{\Delta}{h^2} $}
\BlankLine
\caption{Numerical differentiation of the second order of the trial wavefunction on a system $\mathbf{R}$}\label{alg:nd}
\end{algorithm} 

Since the derivative involves three function calls for each particle the numerical derivative will obviously be quite computationally expensive. It is noted  that the differentiation could be substantially optimized from the version included in the code, but is outside the scope of this project. 


\subsection{One-body densities}
The one body density can be defined in terms of an arbitrary position vector $\mathbf{r}_1$ as 

\begin{equation}\label{eq:obd}
\rho(\mathbf{r}_1) = \int d\mathbf{r}_2 \ldots d\mathbf{r}_N |\Psi(\mathbf{r}_1, \mathbf{r}_2, \ldots, \mathbf{r}_N)|^2
\end{equation}

It's a useful size to compute since it relates the number of particles in the simulation to the density by way of equation \ref{eq:obd_N}

\begin{equation}\label{eq:obd_N}
N = \int d\mathbf{r}\rho(\mathbf{r})
\end{equation}

It also provides a complete picture of the relative distribution of particles. Which can be used to visually identify the ground state\footnote{In the odd case where the computation of $\frac{\partial \langle E_L (\alpha_i) \rangle}{\partial \alpha_i}$ for variational optimization is unavailable or too cumbersome one could ostensibly use very simple ML algorithms to optimize the variational parameter.}. 