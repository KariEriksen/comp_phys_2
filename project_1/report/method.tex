\section{Method}
Keeping the stuff here for examples. Might even be relevant.
\subsection{Randomization}
Randomizing the transaction factor $\epsilon$, and the picking of financial agents,
\lstinline{agent_one} and \lstinline{agent_two}, was done by initializing the following
random number generators (RNGs):
\begin{lstlisting}
std::random_device rd;
std::mt19937_64 gen(rd());
std::uniform_int_distribution<int> AgentPicker(0,NAgents-1);
std::uniform_real_distribution<double> TransactionFactorGenerator(0.0,1.0);
// Calling RNGs to initialize agents and transaction factor:
agent_one = AgentPicker(gen);
agent_two = AgentPicker(gen);
TransactionFactor = TransactionFactorGenerator(gen);
\end{lstlisting}

\subsection{Conservation of money}
A potential source of money "leaks" in the simulations is if 
\lstinline{agent_one} = \lstinline{agent_two}.
In this case the system would "leak" an amount of money equal to $\epsilon(m_1+m_2)$, propagating
for each transaction where that agent is involved, and for each subsequent instance of the error.
This was handled by a simple test
\begin{lstlisting}
if (agent_one == agent_two){
    continue;
}
\end{lstlisting}
which throws away the transactions where this would happen.



