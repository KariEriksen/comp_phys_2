\section{Discussion}
\section{Non interacting bosons}

We've observed that for non-interactive bosons we are able to find an analytical minimum for the local energy, that we've used as a reference to assert that the VMC machinery indeed produces the correct result. 
To decrease the number of MC-cycles needed to reach a minimum, importance sampling was implemented. 
From figure \ref{fig:1c_mc} we observe that importance sampling seems to not visibly improve convergence. 
The observation of values deviating from the analytical local energy (figure \ref{fig:1c_10}),
which is also different from the minima found in figure \ref{fig:1c_dt01}, tells of larger variations in the implementation than is apparent from the variance in the results
(figures \ref{fig:1c_1},\ref{fig:1c_10},\ref{fig:1c_100},\ref{fig:1c_500}).
It is then reasonable to conclude that our implementation might also affect the rate of convergecen by the same error that leads to the loss of precision in $\expect{E_L}$.

As noted in task 1b the analytical minimum is not in general something that is available to us.
Methods like gradient descent and variations thereof must then be considered. We implemented a GD algorithm that converged fast, and with relatively few MC-cycles, to a stable minimum for $\alpha$. More complex methods were not considered since we knew the problem to be quadratic, and thus well behaved in that the search space only has one minimum. 
As and additional metric on the system we computed the one-body desities which shows a delta like behaviour in the non-interactive regime. The physical interpretation of this is that the particles are all very close to eachother and the origio. The authors did not find any demonstrations of the onebody density for bosons in the BEC in a HO trap in the litterature so comparison weren't readily available. 

We have also shown that there is, unsurprisingly, a large chunk of time lost by doing numerical estimations of the local energy. While trivial to implement and compute it is only useful as a sanity-check for the analytical solution for many body systems where an analytic derivative is available. Both the numeric and analytic simulations were run with the \lstinline{-O3} optimization flag. 

There is also shown to be experimental indications of the hypothesis that the ground state of the ensamble of bosons also is  the one that has the lowest variance in the VMC scheme by presenting the relative error in the measurment of $\expect{E_L}$  


\section{Interacting bosons}
In the case of the elliptical trap with interaction we observe the expectation value of the local energy to increases by a small amount as we increase the number of particles in the system. This is consistent with what we see in Fig. 2 in ref \cite{VMC}. As the density of the gas increases so should its energy. 
We are able to find an optimal value for $\alpha$, not without difficulties though. With an analytic solution with many terms something is bound to go wrong. Also with the amount of time it takes to numerically solve the interactive case it is important know the code produces the correct answers. However we are happy with the results.

The onebody densities of the bosons in the ground state was shown to be distributed according to a normal distribution  \footnote{This should probably be rigorously tested, but we didn't}. Which shows a large effect of the jastrow factor on the density, as the shape is nearly delta-function distributed for the non-interactive case. 
